\documentclass[12pt]{article}
\usepackage{booktabs}
\usepackage{float}
\usepackage{import}
\usepackage{xifthen}
\usepackage{pdfpages}
\usepackage{transparent}

%%%%%%%%%%%%%% LATEX SAMPLE FILE %%%%%%%%%%%%%%%%
% A line which starts with a % sign
% is called a COMMENT. It is IGNORED
% by the LaTeX processor.

% Include math
\usepackage{amsmath,amsthm,amssymb}
% Include links
\usepackage{hyperref}


%%%%%%%%%%%%%  THEOREMS  %%%%%%%%%%%%%%%%%
% Let's define some theorem environments
% To use later in the paper
\theoremstyle{plain} % other options: definition, remark
\newtheorem{theorem}{Theorem}
\newtheorem{lemma}[theorem]{Lemma}
% By including [theorem], the lemma follows the numbering of theorem
% e.g. Thm 1, Lemma 2, Thm 3, Thm 4, \dots
\theoremstyle{definition}
\newtheorem*{definition}{Definition} % the star prevents numbering

% Remarks
\theoremstyle{remark}
\newtheorem{remark}{Remark}


%%%%%%%%%%%%%%  PAGE SETUP %%%%%%%%%%%%%%%%%
% LaTeX has big default margins
% The following sets them to 1in
\usepackage[margin=1.5in]{geometry}

% The following sets up some headers
\usepackage{fancyhdr}
\pagestyle{fancy}
\lhead{An Example LaTeX Document} % Left Header
\rhead{\thepage} % Right Header
\cfoot{} % Center Foot (empty)
\newcommand{\half}{\frac{1}{2}}

\begin{document}

\title{An Example LaTeX Document}
\author{Yuanyao Lin}
\date{\today}
\maketitle 

\begin{abstract}
    A simple little testing document to get accustomed to \LaTeX\ !
\end{abstract}

\tableofcontents

% Start of the new page!
\eject

\begin{quote}
    \textbf{Golden Rule}: Do not do anything manually if you can avoid it.

    Often, \LaTeX\ can do it, and can do it better than you can.
\end{quote}

\section{Testing Sections}
blah blah backslashes 
\[
  \sigma = \delta
\]
\eject

\section{Math stuffs}
\subsection{Inline Math}
\begin{verbatim}
    This is also known as: "\(\)" or "im" with SleepyMalc shortcuts
\end{verbatim}

$e^{i \pi} = -1$ or $\half \cdot 2 = 1$.
\subsection{Display Math}
\begin{verbatim}
    This is also known as: "\[\]" or "dm" with SleepyMalc shortcuts
\end{verbatim}
\[
    \sum_{k=1}^n k^3 = \left( \sum_{k=1}^n k \right)^2.
\]

\eject

\section{Numbered and Bulleted Lists}
\begin{quote}
    This is an introduction to \LaTeX's lists
\end{quote}
\begin{verbatim}
    This is a number list using the "enumerate" and \item environment name
\end{verbatim}
\begin{enumerate}
\item yay i am in in a list
\item Nested as well 
    \begin{enumerate}
        \item nested enumerate test.
    \end{enumerate}
\end{enumerate}

\begin{verbatim}
    This is a bullet list using the "itemize" environment name
\end{verbatim}    
\begin{itemize}
    \item HELO
\end{itemize}


\end{document}
